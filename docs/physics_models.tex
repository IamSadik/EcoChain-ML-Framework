\documentclass{article}
\usepackage{amsmath}

\begin{document}

\section{Physics-Based Renewable Energy Models}

This document details the mathematical models used to convert raw meteorological data (from NASA POWER) into estimated power generation metrics for the EcoChain framework.

\subsection{Photovoltaic (Solar) Power Generation}

The solar power output $P_{pv}$ is calculated using the following model:

\begin{equation}
    P_{pv} = \eta_{pv} \cdot A_{pv} \cdot G_{irr} \cdot (1 - \beta (T_{cell} - T_{ref}))
\end{equation}

Where:
\begin{itemize}
    \item $P_{pv}$: Power output (Watts)
    \item $\eta_{pv}$: Module efficiency (assumed 18\%)
    \item $A_{pv}$: Total array area ($m^2$)
    \item $G_{irr}$: Global Horizontal Irradiance ($W/m^2$) [Source: \texttt{ALLSKY\_SFC\_SW\_DWN}]
    \item $\beta$: Temperature coefficient of power (typically 0.004 $^\circ C^{-1}$)
    \item $T_{ref}$: Reference temperature ($25^\circ C$)
\end{itemize}

The cell temperature $T_{cell}$ is estimated from ambient temperature and irradiance:

\begin{equation}
    T_{cell} = T_{amb} + \left( \frac{NOCT - 20}{800} \right) \cdot G_{irr}
\end{equation}

Where:
\begin{itemize}
    \item $T_{amb}$: Ambient air temperature ($^\circ C$) [Source: \texttt{T2M}]
    \item $NOCT$: Nominal Operating Cell Temperature ($45^\circ C$)
\end{itemize}

\subsection{Wind Turbine Power Generation}

The wind power output $P_{wind}$ is governed by the kinetic energy of the air mass:

\begin{equation}
    P_{wind} = \frac{1}{2} \rho A_{rotor} v^3 C_p(\lambda, \theta)
\end{equation}

Where:
\begin{itemize}
    \item $\rho$: Air density ($kg/m^3$)
    \item $A_{rotor}$: Swept area of the rotor ($m^2$)
    \item $v$: Wind speed at hub height ($m/s$)
    \item $C_p$: Power coefficient (Betz limit theoretical max is 0.593, practical max $\approx 0.4$)
\end{itemize}

\subsubsection{Air Density Correction}
Air density $\rho$ is calculated using the ideal gas law:

\begin{equation}
    \rho = \frac{P_{atm}}{R_{specific} \cdot T_K}
\end{equation}

Where:
\begin{itemize}
    \item $P_{atm}$: Atmospheric pressure (Pa) [Source: \texttt{PS} (converted from kPa)]
    \item $R_{specific}$: Specific gas constant for dry air ($287.058 J/(kg \cdot K)$)
    \item $T_K$: Temperature in Kelvin ($T_{amb} + 273.15$)
\end{itemize}

\subsubsection{Wind Speed Height Correction}
Since wind speed is provided at 10m ($v_{10}$) or 50m ($v_{50}$) and turbines operate higher, the Power Law is used:

\begin{equation}
    v_{hub} = v_{ref} \cdot \left( \frac{h_{hub}}{h_{ref}} \right)^\alpha
\end{equation}

Where $\alpha$ is the shear exponent (typically 0.143 for open terrain).

\subsubsection{Operational Constraints}
The theoretical wind power is constrained by the turbine's operational limits:

\begin{equation}
    P_{final} = 
    \begin{cases} 
    0 & v_{hub} < v_{cut-in} \text{ or } v_{hub} > v_{cut-out} \\
    P_{wind} & v_{cut-in} \le v_{hub} < v_{rated} \\
    P_{rated} & v_{hub} \ge v_{rated}
    \end{cases}
\end{equation}

Where:
\begin{itemize}
    \item $v_{cut-in}$: Minimum speed to generate power (3.0 m/s)
    \item $v_{rated}$: Speed at which rated power is reached (12.0 m/s)
    \item $v_{cut-out}$: Maximum safe operating speed (25.0 m/s)
\end{itemize}

\end{document}

